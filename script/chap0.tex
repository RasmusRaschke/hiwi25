\chapter{Mathematical Introduction}
\label{chap:matint}
\vspace*{-0.9cm}

% \startcontents[chapters]
% \printcontents[chapters]{}{1}{}

We start with the necessary mathematical theory. If the reader already feels comfortable with the theory of smooth manifolds, variational calculus on aforementioned spaces, (Lie) groups as well as autonomous differential equations, they may skip this chapter. We assume basic knowledge of linear algebra and calculus.

\section{Group Theory}
We need some fundamental group theory. From here on, all vector spaces are finite-dimensional.
\begin{definition}[Group]
\marginnote{One should think of groups in terms of symmetries: A symmetry of an abstract object can be thought of as a collection of operations leaving said object invariant.}
A \textbf{Group} is a pair $(G, \circ)$ consisting of a set $G$ and an operation $\circ$ such that the following axioms are satisfied:
\begin{itemize}
    \item $\forall a,b,c \in G: a \circ (b \circ c) = (a \circ b) \circ c$
    \item $\forall a \in G \, \exists a^{-1} \in G: a \circ a^{-1} = 1$
    \item $\exists e \in G \, \forall a \in G: e \circ a = a$
\end{itemize}
If the operation is commutative, we call the group \textbf{abelian}.
\end{definition}
There are plenty of examples of groups important in physics:
\begin{eg}
\marginnote{
        Note that every vector space over $\mathbb{R}$ has a basis. Choosing one and representing automorphisms as $n \times n$-matrices yields the usual group structure by matrix multiplication. This also clearly demonstrates that groups of linear maps cannot be abelian in general.}
    Let $V$ be a real vector space. The automorphisms $V \to V$ consitute a group under composition, the \emph{general linear group} of dimension $n$, $\GL_n(V)$.
    There are several important subgroups of $\GL(\mathbb{R}^n)$ we will heavily use later. Let \[\langle \cdot, \cdot \rangle: \mathbb{R}^n \to \mathbb{R}\] be the euclidean scalar product on $\mathbb{R}^n$ and $T \in \GL(\mathbb{R}^n)$. We define the \emph{orthogonal group} as
\[\O(n):=\{Q \in \GL(\mathbb{R}^n) \mid QQ^top = Q^\top Q = \id\} \leq \GL(\mathbb{R}^n).\]  
Further restricting our attention to orthogonal automorphisms with unit determinant yields the \emph{special orthogonal group} \[
    \SO(n) := \{R \in \O(n) \mid \det R = 1\} \leq \O(n).
.\] 
\end{eg}
In physics, we are usually concerned with how a certain group acts on a physical system. For that, we need actions and representations.
\begin{definition}[Group Action]
    Let $G$ be a group with identity $e$ and $M$ be a set. A \textbf{right-action} of $G$ on $M$ is a map \[
    \alpha: G \times X \to X
    .\] 
    such that:
    \begin{itemize}
        \item $\alpha(e,x) = x$
        \item $\alpha(h, \alpha(g,x))=\alpha(gh,x)$
    \end{itemize}
    is satisfied. We write $G \acts M$ and $\alpha(g,x)=:g.x$ for short.
\end{definition}
\begin{definition}[Representation]
    Let $V$ be a real vector space and $G$ be a group. A \textbf{real $G$-representation} is a group homomorphism \[
    \rho: V \to \GL(V)
    .\] 
    We call $V$ \textbf{representation space} and $\deg V$ the degree of the representation.
    
\end{definition}
\section{Differential Geometry}
\subsection*{Smooth Manifolds and Tangent Spaces}
From now on, we use the Einstein summation convention.
\begin{definition}[Topological Manifold]
    A \textbf{topological $n$-manifold} is a topological space $M$ such that:
    \begin{itemize}
        \item $M$ has the Hausdorff property.
        \item $M$ is second-countable.
        \item $M$ is locally euclidean: For every $p \in M$ there is an open neighbourhood $U \subseteq M$ of $p$ and a homeomorphism $\phi: U \to V$ such that $V \subseteq \mathbb{R}^n$ is open. We call $(\phi, U)$ a \textbf{chart} on $M$.
    \end{itemize}
\end{definition}
Since we want to apply the formalism in concrete examples, we will often work in local coordinates. Note that with the standard basis of $\mathbb{R}^n$, we can write a chart map as 
\[
\phi(p)=(\phi_1(p), \dots, \phi_n(p)) =: (x^1(p), \dots, x^n(p))
.\] where $x^i: M \to \mathbb{R}$ are the chart components. Since the charts are homeomorphisms, they can be locally inverted to a map $\phi^{-1}: V \to U$, in which case we call $\phi^{-1}$ a \emph{local parametrization} of $M$.
\begin{eg}
    \begin{itemize}
        \item Trivially, the real euclidean space $\mathbb{R}^n$ is an $n$-dimensional smooth manifold with the identity $\id$ being a global chart.
        \item The $n$-sphere $\mathbb{S}^n$ is a smooth manifold with local charts given by projection onto the coordinate axes.
    \item The $n$-torus \[T^n = \underbrace{\mathbb{S} \times \cdots \times \mathbb{S}}_{n\text{ times}}\] is a smooth $n$-manifold. In the case of $T^2$, one obtains local charts easily as the $2$-torus is a surface of revolution. 
    \end{itemize}
\end{eg}
We need the notion of smoothness on manifolds.
\begin{definition}[Smooth Manifold]
    \marginnote{Note that with this, we get a notion of smooth maps on abstract manifolds: If $M,N$ are smooth manifolds and $F: M \to N$ is a map, we call it smooth if for every $p \in M$ there is a chart $(U,\phi)$ with $p \in U$ and a smooth chart $(V, \psi)$ with $F(p) \in V$ such that \[ \psi \circ F \circ \phi^{-1}: \mathbb{R}^m \to \mathbb{R}^n\] is smooth in the usual sense.}
    A \textbf{maximal atlas} for a topological manifold $M$ is a collection $\mathfrak{A}$ of charts of $M$ such that every $p \in M$ is contained in some chart and $\mathfrak{A}$ is not properly contained in some other atlas. We call $\mathfrak{A}$ smooth if for any two charts $(U,\phi),(V,\psi) \in \mathfrak{A}$ we have either $U \cap V = \emptyset$ or \[
        \phi^{-1} \circ \psi: V \to U
    .\] is a smooth ($\mathcal{C}^\infty$) map, called \textbf{chart transition map}. A \textbf{smooth manifold} is a pair $(M, \mathfrak{A})$ such that $M$ is a topological manifold and $\mathfrak{A}$ is a maximal smooth atlas.
\end{definition}
There are many examples important to physics:
\begin{eg}
    \begin{itemize}
        \item The $\mathbb{R}^n$ obtains a smooth maximal atlas by means of the identity.
        \item The sphere $\mathbb{S}^2$ has a smooth atlas with two charts $\mathbb{S}^2 \setminus N$ and $\mathbb{S}^2 \setminus S$, where $N$ and $S$ are north and south pole, respectively. The chart map is given by the \emph{stereographic projection} $\sigma: \mathbb{S}^2 \setminus N \to \mathbb{R}^2$. 
        \item Any finite-dimensional real normed vector space $V$ is a smooth manifold. The choice of a basis determines an isomorphism $\mathbb{R}^n \cong V$ which we take as global chart. For any other basis, one obtains a basis transformation matrix which is linear, hence a $\mathcal{C}^\infty$ chart transition.
        \item Any open subset $U \subseteq \mathbb{R}^n$ is a smooth manifold on its own with the smooth atlas $\{U, \id_U\}$. We call such a manifold an \emph{open submanifold} of $\mathbb{R}^n$.
    \end{itemize}
\end{eg}
With smooth maps on manifolds, we are already able to talk about curves or paths on such spaces. However, in physics one is especially interested in velocity and acceleration. For this, we will need two additional structures: The tangent bundle, which comes naturally with every smooth manifold, and a metric to measure distances and angles.
We define a \textbf{curve} on a manifold $M$ to be a smooth map \[
\gamma: I \to M
\] 
where $I \subseteq \mathbb{R}$ is some interval.
\begin{definition}[Tangent Bundle]
    Let $M$ be a smooth manifold and $p \in M$. We declare two smooth curves $\gamma_i: I \to M$, $i \in \{1,2\}$ to be equivalent if for any chart $(U,\phi)$ of $p$ we have 
    \begin{equation}
        \left.\frac{d}{dt} (\phi \circ \gamma_1)\right|_{t=0} = \left.\frac{d}{dt} (\phi \circ \gamma_2)\right|_{t=0}.
    \end{equation}
    \marginnote{
        Note that while the tangent space at a point on an $n$-manifold is an $n$-dimensional real vector space (and hence isomorphic to a copy of $\mathbb{R}^n$), the tangent bundle is a $2n$-dimensional space and indeed also a smooth manifold. However, it is not always possible to identify $TM \cong M \times \mathbb{R}^n$. A point in $TM$ is a pair $(p,v)$, where $p \in M$ and $v \in T_pM$. We also have a natural projection $\pi: TM \to M$, given by $\pi(p,v)=p$.
    }
    
    The space of equivalence classes of such curves is called \textbf{tangent space} at $p$ and denoted $T_pM$. The \textbf{tangent bundle} of $M$ is the disjoint union  \[
        TM := \coprod_{p \in M} T_pM
    .\] 
\end{definition}
Tangent vectors act on smooth functions $f: M \to \mathbb{R}$ by
\[
    v(f)=\left.\frac{d}{dt}\right|_{t=0}(f \circ \gamma)(t)
.\] 
where $\gamma$ represents one path with $\dot{\gamma}|_{t=0}=v$. 
\marginnote{Note that this gives rise to the usual partial derivative of a function $g: \mathbb{R}^n \to \mathbb{R}$ by setting \[
        \frac{\partial g}{\partial x^i}\mid_{y}:=\frac{d}{dt}g(y+te_i)
.\] }
Using the fact that for every $p \in M$ there is a chart $(U, (x^1, \dots, x^n))$ centered around $p$, we are able to pull back the basis of $T_{\phi(p)}\mathbb{R}^n$ to $T_pM$. The basis of $T_{\phi(p)}\mathbb{R}^n$ is given by paths of the form $t \mapsto \phi(p) +te_i$, where $e_i$ is the $i$-th canonical basis vector of $\mathbb{R}^n$. Writing this basis as $(\partial_{x^1}, \dots, \partial_{x^n})$, we can pull back along $\phi^{-1}$ to obtain a basis of $T_pM$, defined as
\[
    \partial_i|_p := \phi^{-1}(\phi(p)+te_i)
.\] This expression acts on a smooth function as above by
\[
    \partial_i|_p f = \left.\frac{\partial(f \circ \phi^{-1})}{\partial x^i}\right|_{\phi(p)}
.\] 
\begin{definition}[Differential]
    Let $F: M \to N$ be a smooth map between smooth manifolds. The \textbf{differential} of $F$ at $p \in M$ is a linear map 
    \[
        dF_p: T_pM \to T_{F(p)}N
    \]
    given by \[
        dF_p([\gamma])=[F \circ \gamma]
    .\] where $[\gamma]$ is an equivalence class of curves through $p$.
\end{definition}
Given local charts $(U, (x^1, \dots, x^m))$ around $p$ and $(V, y^1, \dots, y^n)$ around $F(p)$, the differential can be expressed in local coordinates as
\[
    dF_p (\partial_i) = \partial_i F^j|_p \hat{\partial}_j|_{F(p)}
.\] with $F^j = y^j \circ F$ and $\hat{\partial}_j$ being a basis vector of $T_{F(p)}N$.
We conclude this section with the definition of smooth vector fields.
\begin{definition}[Vector Fields]
    Let $M$ be a smooth manifold. A \textbf{smooth vector field} is a smooth section
    \[
    V: M \to TM
    \] such that $\pi \circ V = \id_M$ with the natural projection $\pi: TM \to M$. The space of smooth sections on $M$ is denoted $\mathfrak{X}(M)$. 
\end{definition} 
    \marginnote{Note that vector fields also act naturally on smooth functions $f: M \to \mathbb{R}$ by means of partial derivation: \[
            (Xf)(p)= X_p f = X^i(p) \, \partial_i|_p(f)
    .\] }
Once more, we desire a local coordinate expression. If $(U, (x^1, \dots, x^n))$ is a local chart, we can use the basis of $T_pM$ defined above for all $p \in U$ to write
\[
X_p = X^i(p)\partial_i|_p
\] 
where $X^i: M \to \mathbb{R}$ are the component functions of $X$. If a collection of vector fields $(X_1, \dots, X_n)$ spans $T_pM$ locally or globally, we call it a \textbf{local or global frame}.
Directing our attention one last time to curves, we can now define curves whose velocity vectors form a given vector field.
\begin{definition}[Integral Curves]
    Let $M$ be a smooth manifolds and $V \in \mathfrak{X}(M)$. An \textbf{integral curve} of $V$ is a smooth curve $\gamma: I \to M$ such that
    \[
        \frac{d}{dt}\gamma(t)=V_{\gamma(t)}
    \] for all $t \in I$. 
\end{definition}
We need this notion of walking along the path prescribed by a vector field later for Hamiltonian phase flows. In general, a flow is the rigorous version of this intuitive idea of following a field:
\begin{definition}[Flow]
    \marginnote{Intuitively, one should think of $t \in \mathbb{R}$ as time. A global flow then shifts the whole manifold along some vector field, e.g. the torus rotates under the global flow of a radial vector field.}
    Suppose $M$ is a smooth manifold with $V \in \mathfrak{X}(M)$ such that a unique global integral curve of $V$ exists for all $p \in M$. Denoting the integral curve at $p$ by $\gamma_p: \mathbb{R} \to M$, a \textbf{global flow} is a continuous action of $\mathbb{R}$ on $M$:
    \begin{align*}
        \theta: M \times \mathbb{R} &\to M\\
        (p, t) &\mapsto \gamma_p(t).
    \end{align*}
\end{definition}
\subsection*{Riemannian Metrics}
To measure distances and angles, we need additional structure on our smooth manifolds. Dual to the tangent bundle of $M$, we introduce the \textbf{cotangent bundle}
\[
    T^\ast M:= \coprod_{p \in M} (T_pM)^\ast
\] which is, as usual in linear algebra, given by the space of linear forms
\[
\omega: TM \to \mathbb{R}
.\] The basis dual to $(\partial_1, \dots, \partial_n)$ will be denoted $(dx^1, \dots, dx^n)$. 
\begin{definition}[Riemmanian Manifolds]
    A \textbf{Riemannian metric} on a smooth manifold $M$ is a smooth, symmetric and positive-definite bilinear form
    \[
    g: TM \times TM \to \mathbb{R}
    .\] The pair $(M,g)$ is called \textbf{Riemannian manifold}.
\end{definition}
\marginnote{This also yields the usual notions obtained with a scalar product on $\mathbb{R}^n$: We have a norm on each $T_pM$ defined by \[
        \|v\|_g := \sqrt{\langle v, v \rangle}
\] and an angle given by \[
\cos \sphericalangle (v,w) := \frac{\langle v,w \rangle}{\|v\| \|w\|} \in [0, \pi]
.\] Since each $g_p$ is non-degenerate, we even have the well-known musical isomorphisms $T_pM \cong T_p^\ast M$. If $X=X^i \partial_i$ is a vector field, we identify it with $X^\flat = g_{ij}X^i dx^j$. A covector field $\omega = \omega_i dx^i$ is identified with $\omega^\sharp = g^{ij} \omega_i \partial_j$ where $g^{ij}=g_{ij}^{-1}$. }
In local coordinates at some $p \in M$, we can use the canonical basis of $T_pM$ to write $g_p: T_pM \times T_p M \to \mathbb{R}$ as
\[
    g_p(v,w) = g_p(v^i \partial_i|_p, w^j \partial_j|_p) =: \langle v^i\partial_i|_p, w^j \partial_j|_p \rangle =  v^i w^j g_{ij}
\] where $g_{ij}=g_p(\partial_i|_p, \partial_j|_p)$ denotes the matrix representation of the metric $g$. The most important definition arising from the existence of a metric function is in our case the arc length.
\begin{definition}[Length]
    Let $(M,g)$ be a smooth Riemannian manifold and $\gamma: I \to M$ be a smooth curve. The \textbf{length} of $\gamma$ is given by the functional
    \[
        L[\gamma]:= \int_I \| \partial_t \gamma(t)\|_g \, dt.
    .\] 
\end{definition}
We will now connect those concepts with physics' formalism.
\section{Lagrangian Mechanics}
\begin{definition}[Configuration Manifold]
    The \textbf{configuration manifold} $Q$ of a system of $N$ point masses consists of all possible positions of those $N$ masses. We call $\dim Q$ the \textbf{degrees of freedom}.
\end{definition}{Configuration Manifold}
\begin{eg}
    \begin{itemize}
        \item The configuration manifold of $N$ free point masses in Euclidean $3$-space is given by $\mathbb{R}^{3N}$ as each particle is free to move in three spacial directions independently.
        \item The two-dimensional double pendulum has the configuration manifold $Q=\mathbb{S}^1 \times \mathbb{S}^1$ as each point can move in one copy of $\mathbb{S}^1$. Note that $\mathbb{S}^1 \times \mathbb{S}^1 \cong T$, so we can think of $Q$ as a $2$-torus.
        \item A rigid rod in Euclidean $2$-space is free to rotate around an axis at its end. The configuration manifold is $Q=\mathbb{R}^2 \times \mathbb{S}^1$ since we need two coordinates $(q^1,q^2)$ to describe the position of the axis of rotation and one angle $q^3 \in [0, 2\pi]$ to describe the rotation itself.
    \end{itemize}
\end{eg}
From classical physics we know very well that positions are not sufficient to characterize a system fully, one also needs velocities.
\begin{definition}[Velocity Phase Space]
    \marginnote{In physical notation, we write a local coordinate system of $TQ$ as $(q^1, \dots, q^n, \dot{q}^1, \dots, \dot{q}^n)$.}
    Let $Q$ be the configuration manifold of a physical system. The \textbf{velocity phase space} of $Q$ is the tangent bundle $TQ$.
\end{definition}
With this complete description of our system, we are ready to define the fundamental quantities needed to solve for the equations of motion.
\begin{definition}[Kinetic and Potential Energy]
    \marginnote{Unless specified otherwise, we use the Euclidean metric for $\mathbb{R}^{3N}$. This means that every submanifold $Q$ is endowed with the restricted Euclidean metric, hence kinetic energy is well-defined.}
    Let $TQ$ be the velocity phase space of a physical system $Q$. The \textbf{kinetic energy} is a quadratic form given locally at $q \in Q$ by
    \begin{align*}
        T: T_qQ &\to \mathbb{R}\\
        v &\mapsto \frac{\|v\|^2}{2}
    \end{align*}
    and the \textbf{potential energy} is some smooth function
    \[
    U: Q \to \mathbb{R}
    .\] 
\end{definition}
Note that in the examples above, many systems are constrained to move on some $m$-dimensional submanifold $Q$ of $\mathbb{R}^{3N}$. We call such constraints \textbf{holonomic} if $Q$ is given by $3N-m$ \emph{independent} equations of the type
\[
    f_k(x^1, \dots, x^{3N})=0
\] with $1 \leq k \leq 3N-m$, $(x^1, \dots, x^{3N})$ being global coordinates of $\mathbb{R}^{3N}$ and $f$ being some smooth function in those coordinates. 
\begin{definition}[Lagrangian]
    Let $Q \subset \mathbb{R}^{3N}$ be the configuration $m$-manifold of $N$ point masses $m_1, \dots, m_N$ with $3N-m=k$ holonomic constraints. The \textbf{Lagrangian} is a smooth function
    \begin{align}
        \mathscr{L}: TQ \to \mathbb{R}.
    \end{align}
    We call $\mathscr{L}$ natural if it can be written as 
    \[
    \mathscr{L}=T-U
    \] for some kinetic energy $T$ and some potential energy $U$. 
\end{definition}
With this, we formulate our two main theorems:
\begin{theorem}[Lagrangian System of Motion]
    Let $Q$ be a configuration space, $TQ$ be its velocity phase space, and $\mathscr{L}: TQ \to \mathbb{R}$ be a Lagrangian. A curve $\gamma: I \to Q$ is a motion of $Q$ if it is an extremal of the \textbf{action functional}
    \begin{align}
        S[\gamma]= \int_I \mathscr{L}(\dot{\gamma}) \, dt.
    \end{align}
\end{theorem}
This should be understood as the principal axiom classical mechanical motion seems to obey to our best understanding. By methodes of variational calculus, one shows that this is equal to the following theorem:
\begin{theorem}[Euler-Lagrange]
    In the situation above, let $(q^1, \dots, q^m)=y(t)$ denote a local coordinate description of a point on a motion of $Q$ and $\mathscr{L}\equiv \mathscr{L}(q,\dot{q})$ denote the Lagrangian in local coordinates. Then the evolution of $\gamma(t)$ with $t$ is such that the \textbf{Euler-Lagrange equations}
    \begin{align}
        \frac{d}{dt}\frac{\partial \mathscr{L}}{\partial \dot{q}} - \frac{\partial \mathscr{L}}{\partial q}=0
    \end{align}
    are satisfied.
\end{theorem}
\section{Ordinary Differential Equations}

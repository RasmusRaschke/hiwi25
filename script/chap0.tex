\chapter{Mathematical Introduction}
\label{chap:matint}
\vspace*{-0.9cm}

% \startcontents[chapters]
% \printcontents[chapters]{}{1}{}

We start with the necessary mathematical theory. If the reader already feels comfortable with the theory of smooth manifolds, variational calculus on aforementioned spaces, (Lie) groups as well as autonomous differential equations, they may skip this chapter. We assume basic knowledge of linear algebra and calculus.

\section{Group Theory}
We need some fundamental group theory. From here on, all vector spaces are finite-dimensional.
\begin{definition}[Group]
\marginnote{One should think of groups in terms of symmetries: A symmetry of an abstract object can be thought of as a collection of operations leaving said object invariant.}
A \textbf{Group} is a pair $(G, \circ)$ consisting of a set $G$ and an operation $\circ$ such that the following axioms are satisfied:
\begin{itemize}
    \item $\forall a,b,c \in G: a \circ (b \circ c) = (a \circ b) \circ c$
    \item $\forall a \in G \, \exists a^{-1} \in G: a \circ a^{-1} = 1$
    \item $\exists e \in G \, \forall a \in G: e \circ a = a$
\end{itemize}
If the operation is commutative, we call the group \textbf{abelian}.
\end{definition}
There are plenty of examples of groups important in physics:
\begin{eg}
\marginnote{
        Note that every vector space over $\mathbb{R}$ has a basis. Choosing one and representing automorphisms as $n \times n$-matrices yields the usual group structure by matrix multiplication. This also clearly demonstrates that groups of linear maps cannot be abelian in general.}
    Let $V$ be a real vector space. The automorphisms $V \to V$ consitute a group under composition, the \emph{general linear group} of dimension $n$, $\GL_n(V)$.
    There are several important subgroups of $\GL(\mathbb{R}^n)$ we will heavily use later. Let \[\langle \cdot, \cdot \rangle: \mathbb{R}^n \to \mathbb{R}\] be the euclidean scalar product on $\mathbb{R}^n$ and $T \in \GL(\mathbb{R}^n)$. We define the \emph{orthogonal group} as
\[\O(n):=\{Q \in \GL(\mathbb{R}^n) \mid QQ^top = Q^\top Q = \id\} \leq \GL(\mathbb{R}^n).\]  
Further restricting our attention to orthogonal automorphisms with unit determinant yields the \emph{special orthogonal group} \[
    \SO(n) := \{R \in \O(n) \mid \det R = 1\} \leq \O(n).
.\] 
\end{eg}
In physics, we are usually concerned with how a certain group acts on a physical system. For that, we need actions and representations.
\begin{definition}[Group Action]
    Let $G$ be a group with identity $e$ and $M$ be a set. A \textbf{right-action} of $G$ on $M$ is a map \[
    \alpha: G \times X \to X
    .\] 
    such that:
    \begin{itemize}
        \item $\alpha(e,x) = x$
        \item $\alpha(h, \alpha(g,x))=\alpha(gh,x)$
    \end{itemize}
    is satisfied. We write $G \acts M$ and $\alpha(g,x)=:g.x$ for short.
\end{definition}
\begin{definition}[Representation]
    Let $V$ be a real vector space and $G$ be a group. A \textbf{real $G$-representation} is a group homomorphism \[
    \rho: V \to \GL(V)
    .\] 
    We call $V$ \textbf{representation space} and $\deg V$ the degree of the representation.
    
\end{definition}
\section{Differential Geometry}
\subsection*{Smooth Manifolds and Tangent Spaces}
From now on, we use the Einstein summation convention.
\begin{definition}[Topological Manifold]
    A \textbf{topological $n$-manifold} is a topological space $M$ such that:
    \begin{itemize}
        \item $M$ has the Hausdorff property.
        \item $M$ is second-countable.
        \item $M$ is locally euclidean: For every $p \in M$ there is an open neighbourhood $U \subseteq M$ of $p$ and a homeomorphism $\phi: U \to V$ such that $V \subseteq \mathbb{R}^n$ is open. We call $(\phi, U)$ a \textbf{chart} on $M$.
    \end{itemize}
\end{definition}
Since we want to apply the formalism in concrete examples, we will often work in local coordinates. Note that with the standard basis of $\mathbb{R}^n$, we can write a chart map as 
\[
\phi(p)=(\phi_1(p), \dots, \phi_n(p)) =: (x^1(p), \dots, x^n(p))
.\] where $x^i: M \to \mathbb{R}$ are the chart components. Since the charts are homeomorphisms, they can be locally inverted to a map $\phi^{-1}: V \to U$, in which case we call $\phi^{-1}$ a \emph{local parametrization} of $M$.
\begin{eg}
    \begin{itemize}
        \item Trivially, the real euclidean space $\mathbb{R}^n$ is an $n$-dimensional smooth manifold with the identity $\id$ being a global chart.
        \item The $n$-sphere $\mathbb{S}^n$ is a smooth manifold with local charts given by projection onto the coordinate axes.
    \item The $n$-torus \[T^n = \underbrace{\mathbb{S} \times \cdots \times \mathbb{S}}_{n\text{ times}}\] is a smooth $n$-manifold. In the case of $T^2$, one obtains local charts easily as the $2$-torus is a surface of revolution. 
    \end{itemize}
\end{eg}
We need the notion of smoothness on manifolds.
\begin{definition}[Smooth Manifold]
    \marginnote{Note that with this, we get a notion of smooth maps on abstract manifolds: If $M,N$ are smooth manifolds and $F: M \to N$ is a map, we call it smooth if for every $p \in M$ there is a chart $(U,\phi)$ with $p \in U$ and a smooth chart $(V, \psi)$ with $F(p) \in V$ such that \[ \psi \circ F \circ \phi^{-1}: \mathbb{R}^m \to \mathbb{R}^n\] is smooth in the usual sense.}
    A \textbf{maximal atlas} for a topological manifold $M$ is a collection $\mathfrak{A}$ of charts of $M$ such that every $p \in M$ is contained in some chart and $\mathfrak{A}$ is not properly contained in some other atlas. We call $\mathfrak{A}$ smooth if for any two charts $(U,\phi),(V,\psi) \in \mathfrak{A}$ we have either $U \cap V = \emptyset$ or \[
        \phi^{-1} \circ \psi: V \to U
    .\] is a smooth ($\mathcal{C}^\infty$) map, called \textbf{chart transition map}. A \textbf{smooth manifold} is a pair $(M, \mathfrak{A})$ such that $M$ is a topological manifold and $\mathfrak{A}$ is a maximal smooth atlas.
\end{definition}
There are many examples important to physics:
\begin{eg}
    \begin{itemize}
        \item The $\mathbb{R}^n$ obtains a smooth maximal atlas by means of the identity.
        \item The sphere $\mathbb{S}^2$ has a smooth atlas with two charts $\mathbb{S}^2 \setminus N$ and $\mathbb{S}^2 \setminus S$, where $N$ and $S$ are north and south pole, respectively. The chart map is given by the \emph{stereographic projection} $\sigma: \mathbb{S}^2 \setminus N \to \mathbb{R}^2$. 
        \item Any finite-dimensional real normed vector space $V$ is a smooth manifold. The choice of a basis determines an isomorphism $\mathbb{R}^n \cong V$ which we take as global chart. For any other basis, one obtains a basis transformation matrix which is linear, hence a $\mathcal{C}^\infty$ chart transition.
        \item Any open subset $U \subseteq \mathbb{R}^n$ is a smooth manifold on its own with the smooth atlas $\{U, \id_U\}$. We call such a manifold an \emph{open submanifold} of $\mathbb{R}^n$.
    \end{itemize}
\end{eg}
With smooth maps on manifolds, we are already able to talk about curves or paths on such spaces. However, in physics one is especially interested in velocity and acceleration. For this, we will need two additional structures: The tangent bundle, which comes naturally with every smooth manifold, and a metric to measure distances and angles.
We define a \textbf{curve} on a manifold $M$ to be a smooth map \[
\gamma: I \to M
\] 
where $I \subseteq \mathbb{R}$ is some interval.
\begin{definition}[Tangent Bundle]
    Let $M$ be a smooth manifold and $p \in M$. We declare two smooth curves $\gamma_i: I \to M$, $i \in \{1,2\}$ to be equivalent if for any chart $(U,\phi)$ of $p$ we have 
    \begin{equation}
        \left.\frac{d}{dt} (\phi \circ \gamma_1)\right|_{t=0} = \left.\frac{d}{dt} (\phi \circ \gamma_2)\right|_{t=0}.
    \end{equation}
    \marginnote{
        Note that while the tangent space at a point on an $n$-manifold is an $n$-dimensional real vector space (and hence isomorphic to a copy of $\mathbb{R}^n$), the tangent bundle is a $2n$-dimensional space and indeed also a smooth manifold. However, it is not always possible to identify $TM \cong M \times \mathbb{R}^n$. A point in $TM$ is a pair $(p,v)$, where $p \in M$ and $v \in T_pM$. We also have a natural projection $\pi: TM \to M$, given by $\pi(p,v)=p$.
    }
    
    The space of equivalence classes of such curves is called \textbf{tangent space} at $p$ and denoted $T_pM$. The \textbf{tangent bundle} of $M$ is the disjoint union  \[
        TM := \coprod_{p \in M} T_pM
    .\] 
\end{definition}
Tangent vectors act on smooth functions $f: M \to \mathbb{R}$ by
\[
    v(f)=\left.\frac{d}{dt}\right|_{t=0}(f \circ \gamma)(t)
.\] 
where $\gamma$ represents one path with $\dot{\gamma}|_{t=0}=v$. 
\marginnote{Note that this gives rise to the usual partial derivative of a function $g: \mathbb{R}^n \to \mathbb{R}$ by setting \[
        \frac{\partial g}{\partial x^i}\mid_{y}:=\frac{d}{dt}g(y+te_i)
.\] }
Using the fact that for every $p \in M$ there is a chart $(U, (x^1, \dots, x^n))$ centered around $p$, we are able to pull back the basis of $T_{\phi(p)}\mathbb{R}^n$ to $T_pM$. The basis of $T_{\phi(p)}\mathbb{R}^n$ is given by paths of the form $t \mapsto \phi(p) +te_i$, where $e_i$ is the $i$-th canonical basis vector of $\mathbb{R}^n$. Writing this basis as $(\partial_{x^1}, \dots, \partial_{x^n})$, we can pull back along $\phi^{-1}$ to obtain a basis of $T_pM$, defined as
\[
    \partial_i|_p := \phi^{-1}(\phi(p)+te_i)
.\] This expression acts on a smooth function as above by
\[
    \partial_i|_p f = \left.\frac{\partial(f \circ \phi^{-1})}{\partial x^i}\right|_{\phi(p)}
.\] 
\begin{definition}[Differential]
    Let $F: M \to N$ be a smooth map between smooth manifolds. The \textbf{differential} of $F$ at $p \in M$ is a linear map 
    \[
        dF_p: T_pM \to T_{F(p)}N
    \]
    given by \[
        dF_p([\gamma])=[F \circ \gamma]
    .\] where $[\gamma]$ is an equivalence class of curves through $p$.
\end{definition}
Given local charts $(U, (x^1, \dots, x^m))$ around $p$ and $(V, y^1, \dots, y^n)$ around $F(p)$, the differential can be expressed in local coordinates as
\[
    dF_p (\partial_i) = \partial_i F^j|_p \hat{\partial}_j|_{F(p)}
.\] with $F^j = y^j \circ F$ and $\hat{\partial}_j$ being a basis vector of $T_{F(p)}N$.
We conclude this section with the definition of smooth vector fields.
\begin{definition}[Vector Fields]
    Let $M$ be a smooth manifold. A \textbf{smooth vector field} is a smooth section
    \[
    V: M \to TM
    \] such that $\pi \circ V = \id_M$ with the natural projection $\pi: TM \to M$. The space of smooth sections on $M$ is denoted $\mathfrak{X}(M)$. 
\end{definition} 
    \marginnote{Note that vector fields also act naturally on smooth functions $f: M \to \mathbb{R}$ by means of partial derivation: \[
            (Xf)(p)= X_p f = X^i(p) \, \partial_i|_p(f)
    .\] }
Once more, we desire a local coordinate expression. If $(U, (x^1, \dots, x^n))$ is a local chart, we can use the basis of $T_pM$ defined above for all $p \in U$ to write
\[
X_p = X^i(p)\partial_i|_p
\] 
where $X^i: M \to \mathbb{R}$ are the component functions of $X$. If a collection of vector fields $(X_1, \dots, X_n)$ spans $T_pM$ locally or globally, we call it a \textbf{local or global frame}.
Directing our attention one last time to curves, we can now define curves whose velocity vectors form a given vector field.
\begin{definition}[Integral Curves]
    Let $M$ be a smooth manifolds and $V \in \mathfrak{X}(M)$. An \textbf{integral curve} of $V$ is a smooth curve $\gamma: I \to M$ such that
    \[
        \frac{d}{dt}\gamma(t)=V_{\gamma(t)}
    \] for all $t \in I$. 
\end{definition}
We need this notion of walking along the path prescribed by a vector field later for Hamiltonian phase flows. In general, a flow is the rigorous version of this intuitive idea of following a field:
\begin{definition}[Flow]
    \marginnote{Intuitively, one should think of $t \in \mathbb{R}$ as time. A global flow then shifts the whole manifold along some vector field, e.g. the torus rotates under the global flow of a radial vector field.}
    Suppose $M$ is a smooth manifold with $V \in \mathfrak{X}(M)$ such that a unique global integral curve of $V$ exists for all $p \in M$. Denoting the integral curve at $p$ by $\gamma_p: \mathbb{R} \to M$, a \textbf{global flow} is a continuous action of $\mathbb{R}$ on $M$:
    \begin{align*}
        \theta: M \times \mathbb{R} &\to M\\
        (p, t) &\mapsto \gamma_p(t).
    \end{align*}
\end{definition}
\subsection*{Riemannian Metrics}
To measure distances and angles, we need additional structure on our smooth manifolds. Dual to the tangent bundle of $M$, we introduce the \textbf{cotangent bundle}
\[
    T^\ast M:= \coprod_{p \in M} (T_pM)^\ast
\] which is, as usual in linear algebra, given by the space of linear forms
\[
\omega: TM \to \mathbb{R}
.\] The basis dual to $(\partial_1, \dots, \partial_n)$ will be denoted $(dx^1, \dots, dx^n)$. 
\begin{definition}[Riemmanian Manifolds]
    A \textbf{Riemannian metric} on a smooth manifold $M$ is a smooth, symmetric and positive-definite bilinear form
    \[
    g: TM \times TM \to \mathbb{R}
    .\] The pair $(M,g)$ is called \textbf{Riemannian manifold}.
\end{definition}
\marginnote{This also yields the usual notions obtained with a scalar product on $\mathbb{R}^n$: We have a norm on each $T_pM$ defined by \[
        \|v\|_g := \sqrt{\langle v, v \rangle}
\] and an angle given by \[
\cos \sphericalangle (v,w) := \frac{\langle v,w \rangle}{\|v\| \|w\|} \in [0, \pi]
.\] Since each $g_p$ is non-degenerate, we even have the well-known musical isomorphisms $T_pM \cong T_p^\ast M$. If $X=X^i \partial_i$ is a vector field, we identify it with $X^\flat = g_{ij}X^i dx^j$. A covector field $\omega = \omega_i dx^i$ is identified with $\omega^\sharp = g^{ij} \omega_i \partial_j$ where $g^{ij}=g_{ij}^{-1}$. }
In local coordinates at some $p \in M$, we can use the canonical basis of $T_pM$ to write $g_p: T_pM \times T_p M \to \mathbb{R}$ as
\[
    g_p(v,w) = g_p(v^i \partial_i|_p, w^j \partial_j|_p) =: \langle v^i\partial_i|_p, w^j \partial_j|_p \rangle =  v^i w^j g_{ij}
\] where $g_{ij}=g_p(\partial_i|_p, \partial_j|_p)$ denotes the matrix representation of the metric $g$. The most important definition arising from the existence of a metric function is in our case the arc length.
\begin{definition}[Length]
    Let $(M,g)$ be a smooth Riemannian manifold and $\gamma: I \to M$ be a smooth curve. The \textbf{length} of $\gamma$ is given by the functional
    \[
        L[\gamma]:= \int_I \| \partial_t \gamma(t)\|_g \, dt.
    .\] 
\end{definition}
We will now connect those concepts with physics' formalism.
\section{Lagrangian Mechanics}
\subsection*{Calculus of Variations}
The central concept is concerned with a special function, the \emph{Lagrangian}. While its definition may seem abstract and purely mathematical, it will turn out to be central to a formulation of classical physics on par with Newton's axioms.
\begin{definition}[Lagrangian]
    Let $M$ be a smooth manifold with tangent bundle $TM$. A \textbf{Lagrangian} is a smooth function \[
    \mathcal{L}: TM \times \mathbb{R} \to \mathbb{R}
    .\] If $\mathcal{L}$ is constant in the second argument, we call it \textbf{autonomous}. 
\end{definition}
A priori, we do not impose any given form on $\mathcal{L}$. If we choose local coordinates $\Phi=(x^1, \dots, x^n, \partial_1, \dots, \partial_n)$, the Lagrangian can be expressed as \[
    \mathcal{L}(x^1, \dots, x^n, \partial_1, \dots, \partial_n,t):= \mathcal{L} \circ \Phi^{-1}: \mathbb{R}^{2n} \times \mathbb{R} \to \mathbb{R}
.\] 
Bridging between smooth manifolds and physical systems is the concept of configuration manifolds and (velocity) phase spaces as well as the theorem of Euler-Lagrange.
\begin{definition}[Configuration Manifold]
    \marginnote{To be consistent with notational conventions in physics, we will choose the letter $q$ for coordinates on configuration manifolds. Hence, local coordinates are now given as $\phi=(q^1, \dots, q^n)$.}
    The \textbf{configuration manifold} $Q$ of a system of $N$ point masses consists of all possible positions of those $N$ masses in Euclidean $3$-space. We call $\dim Q$ the \textbf{degrees of freedom}.
\end{definition}
With this definition, which will suffice for the systems considered later, all configuration manifolds are some subsets of $\mathbb{R}^{3N}$. However, we aim at descibing configuration manifolds more precisely than that.
\begin{eg}
    \begin{itemize}
        \item The configuration manifold of $N$ free point masses in Euclidean $3$-space is given by $\mathbb{R}^{3N}$ as each particle is free to move in three spacial directions independently.
        \item The two-dimensional double pendulum has the configuration manifold $Q=\mathbb{S}^1 \times \mathbb{S}^1$ as each point can move in one copy of $\mathbb{S}^1$. Note that $\mathbb{S}^1 \times \mathbb{S}^1 \cong T$, so we can think of $Q$ as a $2$-torus.
        \item A rigid rod in Euclidean $2$-space is free to rotate around an axis at its end. The configuration manifold is $Q=\mathbb{R}^2 \times \mathbb{S}^1$ since we need two coordinates $(q^1,q^2)$ to describe the position of the axis of rotation and one angle $q^3 \in [0, 2\pi]$ to describe the rotation itself.
    \end{itemize}
\end{eg}
From classical physics we know very well that positions are not sufficient to characterize a system fully, one also needs velocities.
\begin{definition}[Velocity Phase Space]
    \marginnote{To give a quick overview: We have $\dim TQ = 2 \cdot \dim Q$, so we need twice as much coordinates on $TQ$. Those are given by the $n$ positions $(q^1, \dots, q^n)$ of our system at some time $t$ together with the $n$ velocities $(\dot{q}_1, \dots, \dot{q}_n)$ at the same time.}
    Let $Q$ be the configuration manifold of a physical system. The \textbf{velocity phase space} of $Q$ is the tangent bundle $TQ$.
\end{definition}
We also have to specify what we are looking for. In physics, one is interested in an \emph{equation of motion} for the considered system. To this end, a \textbf{motion} is a smooth curve
\[
\gamma: I \subseteq \mathbb{R} \to Q
.\] 
We think about $I$ as a time interval, so $\gamma$ specifies one evolution of the system $Q$. The one evolution $\gamma$ actually taken by the system is what we call our \emph{equation of motion}. To obtain such $\gamma$, we have to rely on one empirical axiom.
\begin{axiom}[Hamilton's Principle of Extremal Action]
    \marginnote{We do not define what a Lagrangian system is, since this is quite difficult to answer. To our knowledge, it is possible to conjure some Lagrangian yielding equations of motion. However, this is anything but straightforward outside of classical mechanics.}
    Let $Q$ be a configuration space of a Lagrangian physical system with tangent bundle $TQ$ and Lagrangian $\mathcal{L}$. If and only if a given smooth curve $\gamma: I \to Q$ is an extremal of the \textbf{action functional}
    \[
        S[\mathcal{L}] := \int_I \mathcal{L}(\gamma(t), \dot{\gamma}(t),t)\, dt
    ,\] it is a motion of $Q$. 
\end{axiom}
Using this axiom directly to calculate $\gamma$ from a given Lagrangian is quite cumbersome. The following theorem simplifies this procedure a bit.
\begin{theorem}[Euler-Lagrange]
    \marginnote{Note that while we want to apply this in a physical context, those equations are much more general since we only demand of $\mathcal{L}$ to be smooth. A famous other example of these equations is finding minimal surfaces, found in nature as the thin films soapy water forms on geometrical shapes. To make matters more confusing, one can use these equations to minimize the length functional of the previous section, in which case we call the equations \textbf{Euler equations} or, especially if expressed in local coordinates, \textbf{Geodesic equations}.}
    Let $Q$ be a smooth manifold and $\mathcal{L}$ be a Lagrangian on $TQ \times \mathbb{R}$. A smooth curve $\gamma: I \to Q$ extremizes the action functional $S[\gamma]$ if and only if the following set of equations is satisfied for all $t \in I$:
    \begin{equation}
        \frac{d}{dt} \left( \frac{\partial \mathcal{L}}{\partial \dot{\gamma}^i} \right) - \frac{\partial \mathcal{L}}{\partial \gamma^i} = 0
    \end{equation}
    where $\gamma_i$ are the components of $\gamma$ in some local chart \emph{spanning $TM$}. These differential equations are called \textbf{Lagrange Equations of the second kind}.
\end{theorem}
Since our treatment of the novel physical system relies on this theorem, we want to prove it. We are, however, using a lemma without a proof.
\begin{lemma}[Fundamental Lemma of Variational Calculus]
    Let $f: (a,b) \to \mathbb{R}$ be continuous. If 
    \[
        \int_a^b f(t) \nu (t) \, dt = 0
    \] is satisfied for all smooth functions $\nu: (a,b) \to \mathbb{R}$ with compact support, then $f \equiv 0$. 
\end{lemma}
\begin{proof}
    Since we work with a locally trivializing chart of $TQ$ (meaning the coordinates are a full orthogonal system), the equation $\delta S[\mathcal{L}]=0$ has to hold for each component $\gamma^i$. Hence, all components $\gamma^i$ have to satisfy this equation separately, reducing the problem to one dimension. Therefore, take a path $\gamma: (t_0,t_1) \to \mathbb{R}$ (we read $\gamma$ directly in local coordinates). A variation of $\gamma$ with fixed endpoints is a smooth function \[
    \delta \gamma: (t_0,t_1) \to \mathbb{R}
\] with $\delta \gamma (t_0) = \delta \gamma (t_1)=0$. If $S$ has an extremum at $\gamma$, the functional derivative vanishes for $\gamma$. Using this, we calculate:
\begin{align*}
    0 &= \delta S[\gamma] = \left.\frac{d}{d \epsilon}\right|_{\epsilon = 0} \int_{t_0}^{t_1} \mathcal{L}(\gamma+ \epsilon \cdot \delta \gamma, \dot{\gamma}+ \epsilon \cdot \dot{\delta \gamma},t) \, dt \\
      &= \int_{t_0}^{t_1}\left\{ \frac{\partial \mathcal{L}}{\partial \gamma} \cdot \delta \gamma + \frac{\partial \mathcal{L}}{\partial \dot{\gamma}} \cdot \frac{\partial \delta \gamma}{\partial t} + \frac{d \mathcal{L}}{dt} \right\} \, dt
\end{align*}
Concentrating on the second term, we can integrate by parts:
\begin{align*}
    \int_{t_0}^{t^1} \frac{\partial \mathcal{L}}{\partial \dot{\gamma}} \cdot \frac{\partial \delta \gamma}{\partial t} \, dt &= \frac{\partial \mathcal{L}}{\partial \dot{\gamma}} \cdot \delta \gamma \mid_{t_0}^{t_1} - \int_{t_0}^{t_1} \frac{d}{dt} \frac{\partial \mathcal{L}}{\partial \dot{\gamma}} \cdot \delta \gamma \, dt\\
                                                                                                                              &= -\int_{t_0}^{t_1} \frac{\partial \mathcal{L}}{\partial \dot{\gamma}} \cdot \delta \gamma \, dt.
\end{align*}
The first term vanished as $\delta \gamma$ is a variation with fixed endpoints. Plugging this back in yields
\[
    0 = \int_{t_0}^{t_1} \left\{ \frac{\partial \mathcal{L}}{\partial \gamma} - \frac{d}{dt} \left( \frac{\partial \mathcal{L}}{\partial \dot{\gamma}}\right) \right\} \delta \gamma \, dt
.\] 
Since $\delta \gamma$ is compactly supported by definition, we apply the fundamental theorem of calculus to obtain our claim.\\
For the other direction, the proof can practically be read backwards.
\end{proof}
\subsection*{Physical Lagrangians and Constraints}
After accepting that nature seems to follow a universal extremal principle, we want to explore the different important forms a Lagrangian can take. The easiest examples are such that $\mathcal{L}$ is connected with potential and kinetic energy.
\begin{definition}[Kinetic and Potential Energy]
    \marginnote{Unless specified otherwise, we use the Euclidean metric for $\mathbb{R}^{3N}$. This means that every submanifold $Q$ is endowed with the restricted Euclidean metric, hence kinetic energy is well-defined.}
    Let $TQ$ be the velocity phase space of a physical system $Q$. The \textbf{kinetic energy} is a quadratic form given locally at $q \in Q$ by
    \begin{align*}
        T: T_qQ  \times \mathbb{R} &\to \mathbb{R}\\
        \dot{q} &\mapsto \frac{\|\dot{q}\|^2}{2}
    \end{align*}
    and the \textbf{potential energy} is some smooth function
    \[
    U: Q \times \mathbb{R} \to \mathbb{R}
    .\] If the system is autonomous, $T$ and $U$ are independent of the second argument.
\end{definition}
\begin{definition}[Natural System]
    \marginnote{Again, there are autonomous and non-autonomous natural systems. All systems we consider will be autonomous, so $T$ and $U$ only depend on $t$ indirectly through the time dependence of a path $(\gamma, \dot{\gamma})$.}
    A Lagrangian physical system $Q$ is called \textbf{natural} if the Lagrangian is given by
    \[
        \mathcal{L}(q,\dot{q},t):=T(q,\dot{q},t) - U(q,t)
    \] where $T: TQ \times \mathbb{R} \to \mathbb{R}$ is a kinetic energy and $U: Q \times \mathbb{R} \to \mathbb{R}$ is a potential. 
\end{definition}
\marginnote{If one is not satisfied with this explanation, one can be more physical: Imagine some curve a free particle in a plane can take. Now imagine a constaint as a force acting in one coordinate direction $q^1$ of the local chart $(q^1,q^2)$. If we consider this potential to go to infinity, the particle is forced to move in direction $q^2$ in the limit. It is possible to prove the following proposition by making this idea rigorous.}
Note that in the examples above, many systems are constrained to move on some $m$-dimensional submanifold $Q$ of $\mathbb{R}^{3N}$. We call such constraints \textbf{holonomic} if $Q$ is given by $3N-m$ \emph{independent} equations of the type
\[
    f_k(q^1, \dots, q^{3N},t)=0
\] with $1 \leq k \leq 3N-m$, $(q^1, \dots, q^{3N})$ being global coordinates of $\mathbb{R}^{3N}$ and $f$ being some smooth function in those coordinates. If $f_k$ is independent of $t$, we call it \textbf{scleronomous}, else it is called \textbf{rheonomous}.
\begin{theorem}[Holonomic Systems]
    A physical system with purely holonomic constraints is natural.
\end{theorem}
\begin{eg}
        A free particle with mass $m$ in $\mathbb{R}^{3}$ has the kinetic energy 
            \[
                T(\dot{q}^1,\dot{q}^2,\dot{q}^3)= \frac{m}{2} \delta^{ij} \dot{q}_i \dot{q}_j  
            .\] Its Lagrangian is just $\mathcal{L}=T$. The Lagrange equations of the second kind read as
            \[
                0 = \frac{d}{dt}\left( \frac{\partial T}{\partial \dot{\gamma}^i}\right) = m\dot{\gamma}^i.
        \]
            Given a set of initial conditions $\gamma^i(t_0)=A_i$ and $\dot{\gamma}^i(t_0)=B_i$, we obtain the motion \[
            \gamma^i(t) = A_it+B_i
            .\] 
        
\end{eg}
\subsection*{Conservation Laws}
One huge advantage over the Newtonian formalism is that the Lagrangian directly provides means to connect symmetries of a given system with conservation laws which aid massively in solving for the equations of motion.
\begin{definition}[Symmetry]
    Let $Q$ be a configuration manifold with Lagrangian $\mathcal{L}$. We say that $Q$ admits a \textbf{symmetry} if there is a one-parameter group of diffeomorphisms \[
    \Phi: Q \times  \mathbb{R} \times \mathbb{R} \to M
    \] such that \[
    \mathcal{L}(d\varphi^s (\mathbf{q}, \dot{\mathbf{q}},t))=\mathcal{L}(\mathbf{q},\dot{\mathbf{q}},t)
    \] for all $s \in \mathbb{R}$,
    where we write $\varphi^s(\mathbf{q},t):=\Phi(\mathbf{q},t,s)$.  
\end{definition}
\begin{eg}
    The most common examples of symmetries are spacial translation \[
    \varphi^s(q^1,\dots, q^i, \dots,q^n,t)=(q^1, \dots, q^i+s, \dots, q^n,t)
    \] and temporal translation
    \[
    \varphi^s(\mathbf{q},t)=(\mathbf{q},t+s)
    .\] 
\end{eg}
The celebrated theorem of Emmy Noether will connect symmetries to so-called first integrals. If we have a system of ordinary differential equations
\[
    f(x(t),\dot{x}(t),t)=0
\] with solutions $x: I \subseteq \mathbb{R} \to \mathbb{R}^n$, a \textbf{first integral} of this equation is a smooth function $\mathcal{I}: \mathbb{R}^n \times \mathbb{R} \to \mathbb{R}$ such that
\[
\frac{d}{dt}\mathcal{I}(x(t),t)=0
\] for any solution $x$. This means that first integrals are constant along solutions of Lagrangian equations. Their importance arises from the fact that a first integral can always be integrated to reduce the number of unknowns by $1$. 
\begin{theorem}[Noether]
    If the Lagrangian system $(Q,\mathcal{L})$ admits a symmetry $\Phi$, then the Lagrangian equations have a first integral \[
    \mathcal{I}: TM \to \mathbb{R}
    \] given in local coordinates by \[
    \mathcal{I}(\mathbf{q}, \dot{\mathbf{q}},t)=\frac{\partial \mathcal{L}}{\partial \dot{\mathbf{q}}} \left. \frac{d \Phi(\mathbf{q})}{ds} \right|_{s=0}
    .\] 
\end{theorem}
\begin{eg}
    We consider the case of translational invariance along some axis. Take $N$ point masses $m_i$ in $\mathbb{R}^3$ with $N$ constraints $f_j(\mathbf{q}_j)=0$ and a potential $U=U(\mathbf{q})$ which admits translation in $q_j^1$-direction. The Lagrangian is given by
    \[
        \mathcal{L}=\sum_i \frac{m_i}{2} \delta^{jk} \dot{q}_{ij} \dot{q}_{ik} + U(\mathbf{q})
    .\] The symmetry is given by
    \[
        \varphi^s: q_{i1} \mapsto q_{i1} + s
    .\] The first integral is
    \[
        \mathcal{I}=\sum_i m_i \dot{q}_{i1} = \sum_i p_{i1}
    \] where $p_{i1}$ is the mechanical momentum of the $i$-th particle in direction of $q^1$. This shows that translational symmetry corresponds to conservation of linear, mechanical momentum.
\end{eg}
\subsection*{Non-holonomic Systems}
We want to conclude the section on Lagrangian systems with a short treatment of the case of non-holonomic constraints. Looking back at our system with $3N-m$ degrees of freedom, we use the $m$ holonomic constraints $f_k(\mathbf{q},t)=0$ to eliminate $m$ coordinates by integration. This yields the orthogonal trivialization of $TQ$ which we needed to justify that the coordinate components in the Euler-Lagrange equation vanish independently of each other. Now, we are confronted with $m$ non-holonomic constraints \[
g_k(\mathbf{q}, \dot{\mathbf{q}}, \ddot{\mathbf{q}}, \dots, t)=0
\] which employ a dependence of the coordinates on one another without being integrable. We will not be able to solve this problem in general, but there are noteworthy special cases.\\
If we assume that the constraints are still holonomic but we did not integrate them and therefore still have $\dim Q$ coordinates and $m$ first integrals, we can use Lagrange multipliers to save our principle of extremal action.
\begin{theorem}[Lagrangian Multipliers]
    \marginnote{This notion of a force of constraint fits nicely with our ealier intuitive explanation that constraints can be considered as very strong force fields.}
    Let $(Q, \mathcal{L})$ be a Lagrangian system with $m$ holonomic constraints $f(\mathbf{q},t)=0$. In this case, a smooth path $\gamma: I \to Q$ is an extremal of $S$ if and only if the \textbf{Lagrange Equations of the First Kind}
    \[
        \frac{\partial \mathcal{L}}{\partial \gamma^i} - \frac{d}{dt} \left( \frac{\partial \mathcal{L}}{\partial \dot{\gamma}_i}\right) + \sum_{k=1}^m \lambda_k(t) \frac{\partial f_k}{\partial \gamma^i}=0
    \] are satisfied, where the $\lambda_k(t)$ are Lagrangian multipliers. The term 
    \[
        Q_i = \sum_{k=1}^m \lambda_k(t) \frac{\partial f_k}{\partial \gamma^i}
    \] is called \textbf{force of constraint} or \textbf{generalized force}. 
\end{theorem}
We can apply this equation for a very specific set of non-holonomic constraints.
\begin{definition}[Semi-Holonomic Constraint]
    Let $(Q, \mathcal{L})$ be a Lagrangian system with $m$ constraints of the form 
    \[
        \omega_k = a_{ik}(\mathbf{q},\dot{\mathbf{q}},t) \, dq^i + a_{k0}(\mathbf{q},\dot{\mathbf{q}},t) \, dt = 0
    \] where $1 \leq k \leq m$ and the $\omega_k$ are Pfaffian forms. In this case, the constraints are called \textbf{semi-holonomic}.  
\end{definition}
Unfortunately, this is not enough to apply the Lagrange equations of the first kind.
\begin{theorem}[Integrable Semi-Holonomic Constraints]
    A Langrangian system $(Q, \mathcal{L})$ with a semi-holonomic constraint $\omega$ is equivalent to a holonomic system if and only if there is an integrating factor $\lambda \neq 0$ and a one-form $\eta$ such that
    \[
    \lambda \cdot \omega + f \cdot \eta = df,
    \] where $f$ is the holonomic constraint and $df$ is the exterior derivative of $f$. 
\end{theorem}
\begin{eg}
    A simple example is a hoop rolling down a two-dimensional incline of height $h$ and angle $\alpha$ without slipping. The obvious choice of generalized coordinates are the distance from the starting point and the rotational angle of the hoop, so our first guess at the configuration manifold is $\mathbb{R} \times \mathbb{S}^1$. The kinetic energy is the sum of rotational and translational energy
\[
    T= \frac{m}{2}(\|\dot{q}\|^2+R^2 \|\dot{\theta}\|^2)
\] and the potential is dependent on the height, hence also on $q$:
\[
U= (h-q \sin \alpha)mg
.\] The rolling condition reads as
\[
    \dot{q} - R \dot{\theta}=0
\] which is equivalent to the Pfaffian form
\[
0 = dq-Rd\theta
\] and hence semi-holonomic. One can immediately see that this is integrable, but for the sake of demonstration, we use one Lagrangian multiplier. The modified Lagrangian is
\[
    \mathcal{L}(q,\theta,t,\lambda)=\frac{m}{2}(\|\dot{q}\|^2 + R^2 \|\dot{\theta}\|^2) + (q \sin \alpha - h)mg + \lambda (\dot{q}-R\dot{\theta})
\] and the Lagrange equations of the second kind yield
\[
    m\ddot{q}-mg \sin \alpha + \lambda = 0
\] and \[
m R^2 \ddot{\theta}-\lambda R =0
.\]  Together with our equation of constraint $R \ddot{\theta}= \ddot{q}$, we obtain the following family of solutions
\[
\begin{cases}
    \ddot{q}= \frac{g \sin \alpha}{2}\\
    \ddot{\theta} = \frac{g \sin \alpha}{2R}\\
    \lambda = \frac{mg \sin \alpha}{2}
\end{cases}
.\]  
\end{eg}
\section{Ordinary Differential Equations}
This section is just meant to quickly list the differential equations we will encounter later and give methods to solve them. In this, and only in this, chapter we will do separation of variables in ODEs by seemingly separating differentials of the form $\frac{dx}{dt}$.
\subsection*{Autonomous ODE}
\begin{definition}[Second-Order Autonomous ODE]
    A \textbf{second-order autonomous first-order ODE} is of the form
    \begin{equation}
        \ddot{x}(t)=f(x, \dot{x})
    \end{equation}
    where $x: \Omega \subseteq \mathbb{R} \to \mathbb{R}$ is at least twice differentiable and $f: \Omega \times \mathbb{R} \to \mathbb{R}$ is continuous.
\end{definition}
We consider the special case where $f(x,\dot{x})=f(x)$. The substitution $y:=\frac{dx}{dt}$ yields
\[
    \int y \, dy = \int f(x) \, dx \implies v= \pm \sqrt{2 \int f(x) \, dx + C_1}
\] where $C_1 \in \mathbb{R}$ is some constant of integration. Plugging back yields the implicit solution set
\begin{equation}
    \pm \int \frac{dx}{\sqrt{2 \int f(x)\, dx + C_1}} = t +C_2
\end{equation}
with a second constant $C_2 \in \mathbb{R}$. 

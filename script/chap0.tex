\chapter{Mathematical Introduction}
\label{chap:matint}
\vspace*{-0.9cm}

% \startcontents[chapters]
% \printcontents[chapters]{}{1}{}

We start with the necessary mathematical theory. If the reader already feels comfortable with the theory of smooth manifolds, variational calculus on aforementioned spaces, (Lie) groups as well as autonomous differential equations, they may skip this chapter. We assume basic knowledge of linear algebra and calculus.

\section{Group Theory}
We need some fundamental group theory. From here on, all vector spaces are finite-dimensional.
\begin{definition}[Group]
\marginnote{One should think of groups in terms of symmetries: A symmetry of an abstract object can be thought of as a collection of operations leaving said object invariant.}
A \textbf{Group} is a pair $(G, \circ)$ consisting of a set $G$ and an operation $\circ$ such that the following axioms are satisfied:
\begin{itemize}
    \item $\forall a,b,c \in G: a \circ (b \circ c) = (a \circ b) \circ c$
    \item $\forall a \in G \, \exists a^{-1} \in G: a \circ a^{-1} = 1$
    \item $\exists e \in G \, \forall a \in G: e \circ a = a$
\end{itemize}
If the operation is commutative, we call the group \textbf{abelian}.
\end{definition}
There are plenty of examples of groups important in physics:
\begin{eg}
\marginnote{
        Note that every vector space over $\mathbb{R}$ has a basis. Choosing one and representing automorphisms as $n \times n$-matrices yields the usual group structure by matrix multiplication. This also clearly demonstrates that groups of linear maps cannot be abelian in general.}
    Let $V$ be a real vector space. The automorphisms $V \to V$ consitute a group under composition, the \emph{general linear group} of dimension $n$, $\GL_n(V)$.
    There are several important subgroups of $\GL(\mathbb{R}^n)$ we will heavily use later. Let \[\langle \cdot, \cdot \rangle: \mathbb{R}^n \to \mathbb{R}\] be the euclidean scalar product on $\mathbb{R}^n$ and $T \in \GL(\mathbb{R}^n)$. We define the \emph{orthogonal group} as
\[\O(n):=\{Q \in \GL(\mathbb{R}^n) \mid QQ^top = Q^\top Q = \id\} \leq \GL(\mathbb{R}^n).\]  
Further restricting our attention to orthogonal automorphisms with unit determinant yields the \emph{special orthogonal group} \[
    \SO(n) := \{R \in \O(n) \mid \det R = 1\} \leq \O(n).
.\] 
\end{eg}
In physics, we are usually concerned with how a certain group acts on a physical system. For that, we need actions and representations.
\begin{definition}[Group Action]
    Let $G$ be a group with identity $e$ and $M$ be a set. A \textbf{right-action} of $G$ on $M$ is a map \[
    \alpha: G \times X \to X
    .\] 
    such that:
    \begin{itemize}
        \item $\alpha(e,x) = x$
        \item $\alpha(h, \alpha(g,x))=\alpha(gh,x)$
    \end{itemize}
    is satisfied. We write $G \acts M$ and $\alpha(g,x)=:g.x$ for short.
\end{definition}
\begin{definition}[Representation]
    Let $V$ be a real vector space and $G$ be a group. A \textbf{real $G$-representation} is a group homomorphism \[
    \rho: V \to \GL(V)
    .\] 
    We call $V$ \textbf{representation space} and $\deg V$ the degree of the representation.
    
\end{definition}
\section{Configuration Manifolds and Lie groups}

\begin{definition}[Topological Manifold]
    A \textbf{topological $n$-manifold} is a topological space $M$ such that:
    \begin{itemize}
        \item $M$ has the Hausdorff property.
        \item $M$ is second-countable.
        \item $M$ is locally euclidean: For every $p \in M$ there is an open neighbourhood $U \subseteq M$ of $p$ and a homeomorphism $\phi: U \to V$ such that $V \subseteq \mathbb{R}^n$ is open. We call $(\phi, U)$ a \textbf{chart} on $M$.
    \end{itemize}
\end{definition}
Since we want to apply the formalism in concrete examples, we will often work in local coordinates. Note that with the standard basis of $\mathbb{R}^n$, we can write a chart map as 
\[
\phi(p)=(\phi_1(p), \dots, \phi_n(p)) =: (x^1(p), \dots, x^n(p))
.\] where $x^i: M \to \mathbb{R}$ are the chart components. Since the charts are homeomorphisms, they can be locally inverted to a map $\phi^{-1}: V \to U$, in which case we call $\phi^{-1}$ a \emph{local parametrization} of $M$.
\begin{eg}
    \begin{itemize}
        \item Trivially, the real euclidean space $\mathbb{R}^n$ is an $n$-dimensional smooth manifold with the identity $\id$ being a global chart.
        \item The $n$-sphere $\mathbb{S}^n$ is a smooth manifold with local charts given by projection onto the coordinate axes.
    \item The $n$-torus \[T^n = \underbrace{\mathbb{S} \times \cdots \times \mathbb{S}}_{n\text{ times}}\] is a smooth $n$-manifold. In the case of $T^2$, one obtains local charts easily as the $2$-torus is a surface of revolution. 
    \end{itemize}
\end{eg}
We need the notion of smoothness on manifolds.
\begin{definition}[Smooth Manifold]
    \marginnote{Note that with this, we get a notion of smooth maps on abstract manifolds: If $M,N$ are smooth manifolds and $F: M \to N$ is a map, we call it smooth if for every $p \in M$ there is a chart $(U,\phi)$ with $p \in U$ and a smooth chart $(V, \psi)$ with $F(p) \in V$ such that \[ \psi \circ F \circ \phi^{-1}: \mathbb{R}^m \to \mathbb{R}^n\] is smooth in the usual sense.}
    A \textbf{maximal atlas} for a topological manifold $M$ is a collection $\mathfrak{A}$ of charts of $M$ such that every $p \in M$ is contained in some chart and $\mathfrak{A}$ is not properly contained in some other atlas. We call $\mathfrak{A}$ smooth if for any two charts $(U,\phi),(V,\psi) \in \mathfrak{A}$ we have either $U \cap V = \emptyset$ or \[
        \phi^{-1} \circ \psi: V \to U
    .\] is a smooth ($\mathcal{C}^\infty$) map, called \textbf{chart transition map}. A \textbf{smooth manifold} is a pair $(M, \mathfrak{A})$ such that $M$ is a topological manifold and $\mathfrak{A}$ is a maximal smooth atlas.
\end{definition}
There are many examples important to physics:
\begin{eg}
    \begin{itemize}
        \item The $\mathbb{R}^n$ obtains a smooth maximal atlas by means of the identity.
        \item The sphere $\mathbb{S}^2$ has a smooth atlas with two charts $\mathbb{S}^2 \setminus N$ and $\mathbb{S}^2 \setminus S$, where $N$ and $S$ are north and south pole, respectively. The chart map is given by the \emph{stereographic projection} $\sigma: \mathbb{S}^2 \setminus N \to \mathbb{R}^2$. 
        \item Any finite-dimensional real normed vector space $V$ is a smooth manifold. The choice of a basis determines an isomorphism $\mathbb{R}^n \cong V$ which we take as global chart. For any other basis, one obtains a basis transformation matrix which is linear, hence a $\mathcal{C}^\infty$ chart transition.
        \item Any open subset $U \subseteq \mathbb{R}^n$ is a smooth manifold on its own with the smooth atlas $\{U, \id_U\}$. We call such a manifold an \emph{open submanifold} of $\mathbb{R}^n$.
    \end{itemize}
\end{eg}
\section{Variational Calculus}

\section{Ordinary Differential Equations}

\usepackage{enumitem}
\newlist{abbrv}{itemize}{1}
\setlist[abbrv,1]{label=,labelwidth=1in,align=parleft,itemsep=0.1\baselineskip,leftmargin=!}

\DeclareMathOperator{\Crit}{Crit}
\newcommand{\Cinfty}{C^\infty}

\newcommand{\stable}[1]{W^s(#1)}
\newcommand{\unstable}[1]{W^u(#1)}
\newcommand{\unstableb}[1]{\overline{W}^u(#1)}

\def\symbolentry#1#2#3{\item[#2] #3}
\def\sort#1{}

\makeatletter
\newcommand{\superimpose}[2]{%
  {\ooalign{$#1\@firstoftwo#2$\cr\hfil$#1\@secondoftwo#2$\hfil\cr}}}
\makeatother

% https://tex.stackexchange.com/questions/134863/command-for-transverse-and-not-pitchfork-as-used-in-guillemin-and-pollack
% \newcommand{\tcap}{\mathrel{\mathpalette\superimpose{{\raise0.15ex\hbox{$\top$}}{\cap}}}}
\newcommand{\tcap}{\pitchfork}

\newcommand\traj[2]{\mathcal M(#1, #2)}
\renewcommand\L[2]{\mathcal L(#1, #2)}
\newcommand\Lb[2]{\overline{\mathcal L}(#1, #2)}

\newcommand\nX[3]{n_{#1}(#2, #3)}
\newcommand\NX[3]{N_{#1}(#2, #3)}
\newcommand\HM[3][]{HM_{#1}(C_\bul(#2), \partial_#3)}
\newcommand\HMf[2][]{HM_{#1}(#2)}

\DeclareMathOperator{\Ind}{Ind}
\DeclareMathOperator{\Rank}{Rank}

\DeclareMathOperator{\codim}{codim}
\DeclareMathOperator{\grad}{grad}

\DeclareMathOperator{\Ker}{Ker}
\renewcommand{\Im}{\operatorname{Im}}

\newcommand\sphere[1]{S^{#1}}
\newcommand\cdisk[1]{B^{#1}}
\newcommand\odisk[1]{D^{#1}}
\newcommand\bul{\bullet}

 % \newcommand{\bigstar}{\mathop{\Huge \mathlarger{\mathlarger{*}}}}



\newcommand{\listofsymbols}{
    \chapter*{List of symbols}
    \begin{abbrv}


        % \symbolentry{U}{$U(\epsilon, \eta)$}{Morse chart}
        \symbolentry{0}{$M \tcap N$}{Transverse intersection}
    \end{abbrv}
}

\newcommand{\tpoinc}[1]{\ensuremath{\mathrm P_{\text{top}}^{#1}}}
\newcommand{\spoinc}[1]{\ensuremath{\mathrm P_{\infty}^{#1}}}
\newcommand{\ppoinc}[1]{\ensuremath{\mathrm P_{\text{PL}}^{#1}}}
\newcommand{\cpoinc}[1]{\ensuremath{\mathrm P_{C}^{#1}}}

\newcommand{\tcob}[1]{\ensuremath{\mathrm H_{\text{top}}^{#1}}}
\newcommand{\scob}[1]{\ensuremath{\mathrm H_{\infty}^{#1}}}
\newcommand{\pcob}[1]{\ensuremath{\mathrm H_{\text{PL}}^{#1}}}
\newcommand{\ccob}[1]{\ensuremath{\mathrm H_{C}^{#1}}}

\newcommand{\mant}{\ensuremath{\textsf{Man}_{\text{top}}}}
\newcommand{\mans}{\ensuremath{\textsf{Man}_{\infty}}}
\newcommand{\manp}{\ensuremath{\textsf{Man}_{\text{PL}}}}

\usepackage{booktabs}
\usepackage{array}
\newcommand{\overview}[7]{
    \smallskip
    \begin{center}
        \begin{tabular}{
                >{\centering\arraybackslash}p{1.3cm}%
                >{\centering\arraybackslash}p{1.3cm}%
                >{\centering\arraybackslash}p{1.3cm}%
                >{}p{0.3cm}%
                >{\centering\arraybackslash}p{1.3cm}%
                >{\centering\arraybackslash}p{1.3cm}%
                >{\centering\arraybackslash}p{1.3cm}}
                \tpoinc{#1} & \ppoinc{#1} & \spoinc{#1} && \tcob{#1} & \pcob{#1} & \scob{#1}\\
                #2   & #3 & #4 & & #5 & #6 & #7
        \end{tabular}
    \end{center}
}

\usepackage[utf8]{inputenc}
\usepackage[T1]{fontenc}
\usepackage{textcomp}

\usepackage{url}

\usepackage[
    sorting=nyt,
    style=alphabetic
]{biblatex}
\addbibresource{bibliography.bib}

\usepackage{hyperref}
\hypersetup{
    colorlinks,
    linkcolor={black},
    citecolor={black},
    urlcolor={blue!80!black}
}
\usepackage[noabbrev]{cleveref}

% Adds Bibliography, ... to Table of Contents
\usepackage[nottoc]{tocbibind}

\usepackage{graphicx}
\usepackage{float}
\usepackage[usenames,dvipsnames,svgnames]{xcolor}

% \usepackage{cmbright}

\usepackage{amsmath, amsfonts, mathtools, amsthm, amssymb}
\usepackage{mathrsfs}
\usepackage{cancel}

% Shortcuts for blackboard bold letters, e.g. \A outputs \mathbb{A}
\newcommand{\A}{\mathbb{A}}
		\newcommand{\B}{\mathbb{B}}
		\newcommand{\C}{\mathbb{C}}
		\newcommand{\D}{\mathbb{D}}
		\newcommand{\E}{\mathbb{E}}
		\newcommand{\F}{\mathbb{F}}
		\newcommand{\G}{\mathbb{G}}
		\renewcommand{\H}{\mathbb{H}}
		\newcommand{\I}{\mathbb{I}}
		\newcommand{\J}{\mathbb{J}}
		\newcommand{\K}{\mathbb{K}}
		\renewcommand{\L}{\mathbb{L}}
		\newcommand{\M}{\mathbb{M}}
		\newcommand{\N}{\mathbb{N}}
		\renewcommand{\O}{\mathbb{O}}
		\renewcommand{\P}{\mathbb{P}}
		\newcommand{\Q}{\mathbb{Q}}
		\newcommand{\R}{\mathbb{R}}
		\renewcommand{\S}{\mathbb{S}}
		\newcommand{\T}{\mathbb{T}}
		\newcommand{\U}{\mathbb{U}}
		\newcommand{\V}{\mathbb{V}}
		\newcommand{\W}{\mathbb{W}}
		\newcommand{\X}{\mathbb{X}}
		\newcommand{\Y}{\mathbb{Y}}
		\newcommand{\Z}{\mathbb{Z}}

\def\do#1{\csdef{c#1}{\mathcal{#1}}}
\docsvlist{A,B,C,D,E,F,G,H,I,J,K,L,M,N,O,P,Q,R,S,T,U,V,W,X,Y,Z}

\def\do#1{\csdef{#1c}{\mathcal{#1}}}
\docsvlist{A,B,C,D,E,F,G,H,I,J,K,L,M,N,O,P,Q,R,S,T,U,V,W,X,Y,Z}

% Shortcuts for suetterlin letters
\def\do#1{\csdef{#1f}{\mathfrak{#1}}}
\docsvlist{A,B,C,D,E,F,G,H,I,J,K,L,M,N,O,P,Q,R,S,T,U,V,W,X,Y,Z}
		\newcommand{\fa}{\mathfrak{a}}
		\newcommand{\fb}{\mathfrak{b}}
		\newcommand{\fc}{\mathfrak{c}}
		\newcommand{\fd}{\mathfrak{d}}
		\newcommand{\fe}{\mathfrak{e}}
		\newcommand{\ff}{\mathfrak{f}}
		\newcommand{\fg}{\mathfrak{g}}
		\newcommand{\fh}{\mathfrak{h}}
		\newcommand{\ffi}{\mathfrak{i}}
		\newcommand{\fj}{\mathfrak{j}}
		\newcommand{\fk}{\mathfrak{k}}
		\newcommand{\fl}{\mathfrak{l}}
		\newcommand{\fm}{\mathfrak{m}}
		\newcommand{\fn}{\mathfrak{n}}
		\newcommand{\fo}{\mathfrak{o}}
		\newcommand{\fp}{\mathfrak{p}}
		\newcommand{\fq}{\mathfrak{q}}
		\newcommand{\fr}{\mathfrak{r}}
		\newcommand{\fs}{\mathfrak{s}}
		\newcommand{\ft}{\mathfrak{t}}
		\newcommand{\fu}{\mathfrak{u}}
		\newcommand{\fv}{\mathfrak{v}}
		\newcommand{\fw}{\mathfrak{w}}
		\newcommand{\fx}{\mathfrak{x}}
		\newcommand{\fy}{\mathfrak{y}}
		\newcommand{\fz}{\mathfrak{z}}
		
% Shortcuts for stylized letters
\def\do#1{\csdef{s#1}{\mathscr{#1}}}
\docsvlist{A,B,C,D,E,F,G,H,I,J,K,L,M,N,O,P,Q,R,S,T,U,V,W,X,Y,Z}
\def\do#1{\csdef{#1s}{\mathscr{#1}}}
\docsvlist{A,B,C,D,E,F,G,H,I,J,K,L,M,N,O,P,Q,R,S,T,U,V,W,X,Y,Z}


% Shortcuts for letters with a bar on top, e.g. \Abar outputs \overline{A}
\def\do#1{\csdef{bar#1}{\overline{#1}}}
\docsvlist{a,b,c,d,e,f,g,i,j,k,l,m,n,o,p,q,r,s,t,u,v,w,x,y,z,A,B,C,D,E,F,G,H,I,J,K,L,M,N,O,P,Q,R,S,T,U,V,W,X,Y,Z}
% \hbar is already defined as the symbol ℏ (reduced Planck constant)

% Shortcuts for boldface letters, e.g. \Ab outputs \textbf{A}
\def\do#1{\csdef{b#1}{\textbf{#1}}}
\docsvlist{a,b,c,d,e,f,g,h,i,j,k,l,m,n,o,q,r,t,u,w,x,y,z,A,B,C,D,E,F,G,H,I,J,K,L,M,N,O,P,Q,R,S,T,U,V,W,X,Y,Z}
% \pb is already defined (by the physics package) as a 2-argument command, denoting the anticommutator or Poisson bracket, e.g. \pb{A,B} yields {A,B}
% \sb is already defined in the LaTeX kernel. This is a fundamental LaTeX command, DO NOT overwrite it!
% \vb is already defined (by the physics package) as a 1-argument command, for boldface text, e.g. \vb{A} yields \textbf{A}

% Shortcuts for letters with a tilde on top, e.g. \Atil outputs \widetilde{A}
\def\do#1{\csdef{til#1}{\widetilde{#1}}}
\docsvlist{a,b,c,d,e,f,g,h,i,j,k,l,m,n,o,p,q,r,s,t,u,v,w,x,y,z,A,B,C,D,E,F,G,H,I,J,K,L,M,N,O,P,Q,R,S,T,U,V,W,X,Y,Z}

%%%Command section%%%


%General
\newcommand{\emp}{\emptyset}
\newcommand{\exc}{\backslash}
\newcommand{\sub}{\subseteq}
\newcommand{\sups}{\supseteq}
\newcommand{\capp}{\bigcap}
\newcommand{\cupp}{\bigcup}
\newcommand{\kupp}{\bigsqcup}
\newcommand{\cappkn}{\bigcap_{k=1}^n}
\newcommand{\cuppkn}{\bigcup_{k=1}^n}
\newcommand{\kuppkn}{\bigsqcup_{k=1}^n}
\newcommand{\cappa}{\bigcap_{\alpha\in I}}
\newcommand{\cuppa}{\bigcup_{\alpha\in I}}
\newcommand{\kuppa}{\bigsqcup_{\alpha\in I}}
\renewcommand{\>}{\rangle}
\newcommand{\<}{\langle}
\newcommand{\id}{\text{id}}

%Analysis
\newcommand{\limk}{\lim_{k\to\infty}}
\newcommand{\limm}{\lim_{m\to\infty}}
\newcommand{\limn}{\lim_{n\to\infty}}
\newcommand{\limx}[1][a]{\lim_{x\to#1}}
\newcommand{\limz}[1][{z_0}]{\lim_{z\to#1}}
\newcommand{\liminfm}{\liminf_{m\to\infty}}
\newcommand{\limsupm}{\limsup_{m\to\infty}}
\newcommand{\liminfn}{\liminf_{n\to\infty}}
\newcommand{\limsupn}{\limsup_{n\to\infty}}
\newcommand{\sumkn}{\sum_{k=1}^n}
\newcommand{\sumk}[1][1]{\sum_{k=#1}^\infty}
\newcommand{\summ}[1][1]{\sum_{m=#1}^\infty}
\newcommand{\sumn}[1][1]{\sum_{n=#1}^\infty}
\newcommand{\dx}{\,dx}
\newcommand{\dy}{\,dy}
\newcommand{\dz}{\,dz}
\newcommand{\dt}{\,dt}
\newcommand{\dmu}{\,d\mu}
\newcommand{\dnu}{\,d\nu}
\newcommand{\graph}{\text{graph}}
\DeclareMathOperator{\cis}{\text{cis}}
\DeclareMathOperator{\supp}{supp}

%Algebra
\newcommand{\quotient}[2]{{\raisebox{.0em}{$#1$}/\raisebox{-.2em}{$#2$}}} %Quotient Space
\newcommand{\invquotient}[2]{{\raisebox{-.2em}{$#1$}\textbackslash\raisebox{.0em}{$#2$}}} %Left Quotient Space
\DeclareMathOperator{\Hom}{\text{Hom}}
\DeclareMathOperator{\ord}{\text{ord}}
\DeclareMathOperator{\Ann}{\text{Ann}}
\DeclareMathOperator{\Gal}{\text{Gal}}
\DeclareMathOperator{\Aut}{\text{Aut}}
\newcommand{\spn}{\trigbraces{\operatorname{span}}}
\newcommand{\acts}{\curvearrowright}
\newcommand{\lnormal}{\trianglelefteq}
\newcommand{\rnormal}{\trianglerighteq}
\newcommand{\Sym}{\text{Sym}}
\newcommand{\Symso}{\text{Sym}^\text{SO}}
%Topology

%Category

%Differential Geometry
\newcommand{\iprod}{\mathbin{\lrcorner}}
\newcommand{\diffm}{\text{Diff}}
\DeclareMathOperator{\Ric}{Ric}
\DeclareMathOperator{\ric}{ric}
\newcommand{\inj}{\text{inj}}
\DeclareMathOperator{\gst}{g_{st}}
\newcommand{\hodge}{{\star}}

% Shortcuts for inverse hyperbolic functions (and other operators with the same structure)
\def\do#1{\csdef{#1}{\trigbraces{\operatorname{#1}}}}
\docsvlist{
    asinh,acosh,atanh,acoth,asech,acsch,
    arsinh,arcosh,artanh,arcoth,arsech,arcsch,
    arcsinh,arccosh,arctanh,arccoth,arcsech,arccsch,
    sen,tg,cth,senh,tgh,ctgh,
    Re,Im,arg,Arg,im,ker,cis
}


%Maintenance
\newcommand{\diff}{\mathcal{D}}
\newcommand{\so}[1]{\text{SO}(#1)}
\newcommand{\der}[1]{\text{Der}_{#1}}
\newcommand{\gl}[2]{\text{GL}(#1, #2)}
%Define behavior of the command with one parameter
\newcommand{\cinfa}[1]{\text{C}^\infty (#1)}
%Define behavior of the command with two parameters
\newcommand{\cinfb}[2]{\text{C}^\infty (#1, #2)}
\newcommand{\fracpart}[1]{\frac{\partial}{\partial #1_i}}
\NewDocumentCommand\cinf{ m g }{
  \IfNoValueTF{#2}{\cinfa{#1}}{\cinfb{#1}{#2}}
}
\newcommand{\sgn}{\text{sgn}}
\newcommand{\diag}{\text{diag}}

\makeatletter
% Redefining the commands \iff (given by LaTeX), \implies and \impliedby (given by amsmath)
% Math mode is automatically enforced, starred version makes the arrows shorter
\renewcommand{\impliedby}{\@ifstar{\ensuremath{\Longleftarrow}}{\ensuremath{\Leftarrow}}} % Corresponding Unicode character: U+21D0 ⇐
\renewcommand{\implies}{\@ifstar{\ensuremath{\Longrightarrow}}{\ensuremath{\Rightarrow}}} % Corresponding Unicode character: U+21D2 ⇒
\renewcommand{\iff}{\@ifstar{\ensuremath{\Longleftrightarrow}}{\ensuremath{\Leftrightarrow}}} % Corresponding Unicode character: U+21D4 ⇔
\makeatother


\usepackage{tikz}
\usepackage{tikz-cd}

% theorems
\usepackage{thmtools}
\usepackage{thm-restate}
\usepackage[framemethod=TikZ]{mdframed}
\mdfsetup{skipabove=1em,skipbelow=0em, innertopmargin=12pt, innerbottommargin=8pt}

\theoremstyle{definition}

\makeatletter

\declaretheoremstyle[headfont=\bfseries\sffamily, bodyfont=\normalfont, mdframed={ nobreak } ]{thmgreenbox}
\declaretheoremstyle[headfont=\bfseries\sffamily, bodyfont=\normalfont, mdframed={ nobreak } ]{thmredbox}
\declaretheoremstyle[headfont=\bfseries\sffamily, bodyfont=\normalfont]{thmbluebox}
\declaretheoremstyle[headfont=\bfseries\sffamily, bodyfont=\normalfont]{thmblueline}
\declaretheoremstyle[headfont=\bfseries\sffamily, bodyfont=\normalfont, numbered=no, mdframed={ rightline=false, topline=false, bottomline=false, }, qed=\qedsymbol ]{thmproofbox}
\declaretheoremstyle[headfont=\bfseries\sffamily, bodyfont=\normalfont, numbered=no, mdframed={ nobreak, rightline=false, topline=false, bottomline=false } ]{thmexplanationbox}

\declaretheoremstyle[headfont=\bfseries\sffamily, bodyfont=\normalfont, numbered=no, mdframed={ nobreak, rightline=false, topline=false, bottomline=false } ]{thmexplanationbox}


\declaretheorem[numberwithin=chapter, style=thmgreenbox, name=Definition]{definition}
\declaretheorem[sibling=definition, style=thmredbox, name=Corollary]{corollary}
\declaretheorem[sibling=definition, style=thmredbox, name=Proposition]{prop}
\declaretheorem[sibling=definition, style=thmredbox, name=Theorem]{theorem}
\declaretheorem[sibling=definition, style=thmredbox, name=Lemma]{lemma}
\declaretheorem[sibling=definition, style=thmbluebox,  name=Example]{eg}
\declaretheorem[sibling=definition, style=thmbluebox,  name=Nonexample]{noneg}
\declaretheorem[sibling=definition, style=thmblueline, name=Remark]{remark}




\declaretheorem[numbered=no, style=thmexplanationbox, name=Proof]{explanation}
\declaretheorem[numbered=no, style=thmproofbox, name=Proof]{replacementproof}
\declaretheorem[style=thmbluebox,  numbered=no, name=Exercise]{ex}
\declaretheorem[style=thmblueline, numbered=no, name=Note]{note}

% \renewenvironment{proof}[1][\proofname]{\begin{replacementproof}}{\end{replacementproof}}

% \AtEndEnvironment{eg}{\null\hfill$\diamond$}%

\newtheorem*{uovt}{UOVT}
\newtheorem*{notation}{Notation}
\newtheorem*{previouslyseen}{As previously seen}
\newtheorem*{problem}{Problem}
\newtheorem*{observe}{Observe}
\newtheorem*{property}{Property}
\newtheorem*{intuition}{Intuition}


\declaretheoremstyle[
    headfont=\bfseries\sffamily\color{RawSienna!70!black}, bodyfont=\normalfont,
    mdframed={
        linewidth=2pt,
        rightline=false, topline=false, bottomline=false,
        linecolor=RawSienna, backgroundcolor=RawSienna!5,
    }
]{todo}
\declaretheorem[numbered=no, style=todo, name=TODO]{TODO}


\usepackage{etoolbox}
\AtEndEnvironment{vb}{\null\hfill$\diamond$}%
\AtEndEnvironment{intermezzo}{\null\hfill$\diamond$}%

% http://tex.stackexchange.com/questions/22119/how-can-i-change-the-spacing-before-theorems-with-amsthm
% \def\thm@space@setup{%
%   \thm@preskip=\parskip \thm@postskip=0pt
% }

\usepackage{xifthen}

\makeatother

% figure support (https://castel.dev/post/lecture-notes-2)
\usepackage{import}
\usepackage{xifthen}
\pdfminorversion=7
\usepackage{pdfpages}
\usepackage{transparent}


\makeatletter
\newif\ifworking
\@ifclasswith{tuftebook}{working}{\workingtrue}{\workingfalse}
\makeatother

\newcommand{\incfig}[2][1]{%
    % \ifworking{\makebox[0pt][c]{\color{gray}{\scriptsize\textsf{#2}}}}\fi%
    \def\svgwidth{#1\textwidth}
    \import{./figures/}{#2.pdf_tex}
}

\newcommand{\fullwidthincfig}[2][0.90]{%
    % \ifworking{\makebox[0pt][l]{\color{gray}{\scriptsize\textsf{#2}}}}\fi%
    \def\svgwidth{#1\paperwidth}
    \import{./figures/}{#2.pdf_tex}
}



\newcommand{\minifig}[2]{%
    \def\svgwidth{#1}%
    \begingroup%
    \setbox0=\hbox{\import{./figures/}{#2.pdf_tex}}%
    \parbox{\wd0}{\box0}\endgroup%
    \hspace*{0.2cm}
}

% %http://tex.stackexchange.com/questions/76273/multiple-pdfs-with-page-group-included-in-a-single-page-warning
\pdfsuppresswarningpagegroup=1

\newcommand\todo[1]{\ifworking {{\color{red}{#1}}} \else {}\fi}
\newcommand\charlotte[1]{\ifworking {{\color{blue}{#1}}} \else {}\fi}

\author{Gilles Castel}



\usepackage{multirow}
\def\block(#1,#2)#3{\multicolumn{#2}{c}{\multirow{#1}{*}{$ #3 $}}}

% \overfullrule=1mm

\newenvironment{myproof}[1][\proofname]{%
  \proof[\rm \bf #1]%
}{\endproof}
